\chapter{Glossar}

\paragraph{Verifikation}
Die Verifikation stellt fest ob die Software mit der vorhandenen Spezifikation übereinstimmt.
\paragraph{Validierung}
Die Validierung bestimmt ob die Software für den Kunden auch wirklich nützlich ist.
\paragraph{Technische Schuld}
Technische Schuld (technical debt) ist ein wichtiges Thema in der Qualitätssicherung. In der Softwareentwicklungen ist damit gemeint, dass ungeschickte/schlechte Lösungen irgendwann besser gelöst werden müssen.
\paragraph{Softwareprozesse}
Eine Abfolge von Tätigkeiten, durch die ein Software-Produkt entsteht.
\paragraph{Vorgehensmodell}
Vereinfachte Beschreibung eines Softwareprozesses
\paragraph{Methode}
Strukturierter Ansatz für die Software-Entwicklung
\paragraph{Software-Engineering}
Software-Engineering ist eine technische Disziplin welche eine Lösung für den Anwender bieten will unter Einsatz von Theorien, Methoden und Werkzeugen und Berücksichtigung von Organisation, Management und Entwicklung. 
\paragraph{ESSENCE Kernel Overview}
In diesem Model gibt es verschiedene Zustände für die Anforderungen und jeder dieser Zustände hat selbst wieder Kriterien welche einem dabei helfen in welchem Zustein ein Projekt ist.
\paragraph{Qualität}
Grad in dem die inhärenten Eigenschaften des Produkts Anforderungen erfüllen.
\paragraph{Anforderungen}
Eine Anforderung ist eine Aussage über die notwendige Beschaffenheit oder Fähigkeit, die ein System oder Systemteile erfüllen oder besitzen muss, um einen Vertrag zu erfüllen oder einer Norm, einer Spezifikation oder anderen, formell vorgegebenen Dokumenten zu entsprechen. \footnote{vgl.: \url{http://de.wikipedia.org/wiki/Anforderung} (12.06.2014)}
\paragraph{Implizite Anforderungen}
Anforderungen welche existieren aber den Stakeholdern nicht bewusst sind.
\paragraph{Nicht explizite Anforderungen}
Stakeholer wissen, dass sie diese Anforderungen gibt aber sie teilen diese nicht mit (sind quasis eh klar).
\paragraph{Objektive Anforderungen}
Vollkommen klar, dass man etwas braucht.
\paragraph{Subjektive Anforderungen}
Vermeintliche Anforderungen, welche nicht wirklich wichtig sind.
\paragraph{Technisches computer-basiertes System}
System welches ausschliesslich aus Soft- und Hardware-Komponenten besteht.
\paragraph{Soziotechnisches System}
System bestehend aus einem oder mehreren technischen Systemen,
den Menschen die es bedienen, den notwendingen Arbeitsprozessen, organisatorischen Richtlinien.
\paragraph{Kritische Systeme}
Systeme, bei dessen Ausfall oder Fehlfunktion groÿen Schaden anrichten kann (wirtschaftliche Verluste, physische Schäden, Gefahr für Gesundheit und Leben von Menschen).
\paragraph{Sicherheitskritisches System}
Schäden an der Umwelt und/oder Gefahr für Gesundheit und Leben von Menschen (Bohrinsel im Golf von Mexiko).
\paragraph{Aufgabenkritisches System}
Aufgaben die ein System erledigen solle werden nicht durchgeführt (z.B. Bank).
\paragraph{Geschäftskritisches System}
Extrem hohe Kosten bzw. signifikante Gewinnausfälle können die Folge eines Systemausfalls sein.
\paragraph{Ethnographische Methode}
Die Grundidee ist das Beobachten ohne einzugreifen.
\paragraph{Review}
Mit dem Review werden Arbeitsergebnisse der Softwareentwicklung manuell geprüft. Jedes Arbeitsergebnis kann einer Durchsicht durch eine andere Person unterzogen werden. \footnote{vgl.: \url{http://de.wikipedia.org/wiki/Review_(Softwaretest)} (12.06.2014)}
\paragraph{Nutzwertanalyseteam}
Bewertet den Nutzen für die Anwender.
\paragraph{Kostenanalyseteam}
Schätzt die Kosten jeder Anforderung
\paragraph{Kontextmodelle}
Definiert die Systemgrenzen des Gesamtsystems und des technischen Systems. Das Gesamtsystem umfasst das technische System und die menschliche Komponente und definiert den Kontext.
\paragraph{Verhaltensmodelle}
Definiert die Abläufe in einem System. Sie können Datenfluss (Datenfocus)- oder Ereignis (Ereignisfocus) - Orientiert geschehen.
\paragraph{Datenmodelle}
Definiert die logische und persistente (lokal, übergreifend mttels Datenbank
und Datenaustausch) Datenstruktur.
\paragraph{Objektorientierte Modellierung}
Vereinigt die Funktionalität von Daten- und Verhaltensmodelle und können Daten, Datenflüsse, Datenstrukturen und Erreignise erfassen.
\paragraph{Strukturierte Methoden}
Detailliert definierte Vorgehensweise bei der SW-Entwicklung.Normalerweise basierend auf einem Satz von Diagrammtypen. Definiert zusätzliche Regeln
und Richtlinien.
\paragraph{Anforderungsmanagementsystem}
Die Anforderungen an ein System ändern sich mit der Zeit, diese können ursprünglich unvollständig sein. Oder es verbessert sich das Verständnis des Problems / es ensteht eine bessere Sicht der Dinge. Aber auch das Umfeld kann sich verändern, z.B. wirtschaftlich, technisch, juristisch ...
\paragraph{Dauerhafte Anforderungen}
Sie sind relativ stabil und sind mit dem Kern der Anwendung verwoben. Oft aus Standardisierten Modellen entnohmen.
\paragraph{Veränderliche Anforderungen}
Anforderungen mit hoher Änderungswahrscheinlichkeit. Können wirtschaftliche Randbedingungen, technische Randbedingungen und gesetzliche Randbedingungen.
\paragraph{Pflichtenheft}
Zusammenstellung der vollständigen und detailierten Benutzeranforderungen und Systemanforderungen (groÿes Problem ist die Vollständigkeit). Darf nach Fertigstellung nicht mehr geändert werder und muss so implementiert werden wie es festgehalten wurde.
\paragraph{Funktionale Anforderungen}
Funktionale Anforderungen unterstützen Definition von Funktionen zur Fehlerprüfung, zur Wiederherstellung im Fehlerfall und Aspekte zum Schutz gegen Systemausfälle.
\paragraph{Nichtfunktionale Anforderungen}
Als nichtfunktionale Anforderungen werden u.a. die Systemzuverlässigkeit und Systemverfügbarkeit festgelegt.
\paragraph{Negativanforderungen}
Negativanforderungen Beschreibung von Verhalten oder Eigenschaften, die das System auf keinen Fall zeigen darf.
\paragraph{Risikomanagement}
Risikomanagement besteht aus Gefahrenbestimmung, Risikoanalyse und Gefahrenklassifizierung, Gefahrenvereinzelung und Festlegung zur Risikominimierung.
\paragraph{Risikoerkennung}
Ziel ist es zu Erkennen von Rsikiken und Gefahren. Probleme die dabei entstehenkönnen sind die Wechselwirkung zwischen Systemkomponenten bzw. die Wechselwirkungen mit der Umwelt.
\paragraph{Risikoanalyse und Risikoklassifizierung}
Analyse von Unfallwahrscheinlichkeit, Schadenswahrscheinlichkeit und Schadeshöhe.
\paragraph{Risikozerlegung}
Ziel es die Ursachen aufzudecken. Dazu können Reviews, Checklisten, Petri-Netze, formale Login und Fehlerbäume verwendet werden.
\paragraph{Risiko-Minimierung}
Ziel ist das Vermeiden des Auftretens von Gefahren. In der Praxis bedeutet dies eine Kombination aller Strategien.
\paragraph{Systemsicherheit}
Hierbei geht es um direkte und indirekte Berodhungen.
\paragraph{Direkte Bedrohung}
Z.B. unbefugtes Eindringen und DOS.
\paragraph{Indirekte Bedrohung}
Z.B. Aufwand um Sicherheitsmassnahmen zu installieren, konfigurieren und aktuell zu halten. Auch inkludiert sind Fehler und Nachlässigkeiten der Systemverwalter und monopolartige Stellungen einiger Hard- und Softwareanbieter da eine weite Verbreitung von Sicherheitslücken sehr wahrscheinlich ist.
\paragraph{Zuverlässigkeit}
Zuverlässigkeit besteht aus Hardware-, Software- und Bedienerzuverlässigkeit.
\paragraph{Formale Spezifikationsmethoden}
Formale Spezifikationstechniken sind gute Ergänzungen, eindeutig und präzise
aber schwer verständlich für den Laien. Erzwingen die frühzeitige Analyse der Systemanforderungen wenn die Fehlerbehebung noch billig ist.
\paragraph{Ausnahmebehandlung}
Wird in manchen Sprachen mangelhaft Unterstützt (Beispiel: C). Kann auch dazu verwendet werden um die Lesbarkeit der Anwendung zu erhöhen und damit können Fehler vermieden werden.
\paragraph{Software-Reengineering}
Dabei handelt es sich um eine Neuentwicklung von Software, meistens eine Umstellung auf neuere Technologien (z.B. von COBOL auf Java).
\paragraph{Testing}
Betrifft die Kernel Alphas Softwaresystem, Requirements und Work. Tests sollen uns helfen Probleme zu entdecken und beheben, die Kunden überzeugen und die Softwarequalität demonstrieren. Dabei ist allerdings zu beachten, dass nur die Anwesenheit von Fehlern getestet werden kann, allerdings nicht ihre Abwesenheit.
\paragraph{Testprozesse}
Ein Testprozess muss geplant und vorbereitet werden. Dazu gehören die Definitionen der Testziele und verwendeten Messungen, sowie die Eingabewerte und Testsuites.
\paragraph{Vollständiges Testen}
Das vollständige Testen ist meist theorethisch möglich, aber nur in den seltensten Fällen wirtschaftlich. Unser Ziel muss es deshalb sein eine möglichst hohe Testabdeckung zu erreichen, und dabei auch noch wirtschaftlich zu bleiben.