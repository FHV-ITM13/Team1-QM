\chapter{Einführung}
Das Hochhalten von Softwarequalität ist eine sehr anspruchsvolle Aufgabe, welche eine gewisse Planung vorraussetzt. Gerade bei Projekten mit hohen Budgets und materiellen Einsätzen werden hohe Anforderungen an das Qualitätsmanagement gesetzt.

Einige bekannte Fehlschläge in der Softwareentwicklung hätten wahrscheinlich mit besseren Qualitätssicherungsmaßnahmen verhindert werden können:

\begin{itemize}
    \item Pioneer 4 verfehlte den Mond
    \item unnötige Mahnungen durch die französische Finanzverwaltung
    \item das Herausgeben von faulen Krediten, was schlussendlich zum Bankencrash geführt hat
    \item Verlust einer Segelyacht im Pazifik
\end{itemize}

\section{Verifikation und Validierung}
Die Begriffe sollten klar getrennt werden, da sie nicht synonym verwendet werden können, und bei Softwareprozessen und Softwarequalität eine gewichtige Rolle spielen.

Die Verifikation stellt fest ob die Software mit der vorhandenen Spezifikation übereinstimmt, wohingegen die Validierung bestimmt ob die Software für den Kunden auch wirklich nützlich ist.

\section{Technical Debt}
Technical Debts sind ein wichtiges Thema in der Qualitätssicherung. Der Begriff wurde aus dem Finanzwesen übernommen (Debt = Schulden). In der Softwareentwicklungen ist damit gemeint, dass ungeschickte Lösungen irgendwann gefixt werden müssen.

Diese technische Schulden kann man bewusst eingehen, um Termine zu halten. Allerdings muss man immer im Hinterkopf behalten, dass man diese Schulden zu einem späteren Zeitpunkt auch wieder bezahlen muss.

Einer der größten Unterschiede zwischen klassischen und agilen Projektmanagementmethoden ist der Umgang mit diesen technischen Schulden. Bei agilen Methoden werden diese Schulden bewusst eingegangen, um ein schnelleres iteratives Vorgehen zu gewährleisten. Bei diesen Methodiken ist allerdings auch die Gefahr einer Überschuldung ungleich höher als bei den klassichen.

Allerdings werden in der Regel auch bei klassischen Methoden technische Schulden verursacht, nämlich in Form von Änderungen in den Anforderungen, auf welche in agilen Methoden besser reagiert werden kann.
